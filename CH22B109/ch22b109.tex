\documentclass{article}
\usepackage{url}

\begin{document}

\section{CH22B109}

In this section, we will explore an equation and its description.

\subsection{Details}

\begin{itemize}
  \item Name: Suprasidh
  \item Roll No: CH22B109
  \item GitHub: Suprxsidh17
\end{itemize}

\subsection{Equation}

The equation we will examine is the Riemann zeta function, denoted by \(\zeta(s)\), which is defined as:

\begin{equation}
    \zeta(s) = \sum_{n=1}^{\infty} \frac{1}{n^s}
\end{equation}

or

\begin{equation}
    \zeta(s) =\frac{1}{1^s}+\frac{1}{2^s}+\frac{1}{3^s}+...
\end{equation}
 
The Riemann Zeta Function is named after the German mathematician Bernhard Riemann and is defined for complex values of \(s\) with a 
real part greater than 1.

\subsection{Description}

The Riemann zeta function is an important mathematical function with wide-ranging applications in number theory, physics, and other 
fields. It provides a way to extend the concept of the sum of positive integer powers to complex numbers.

The equation represents an infinite series, where for each positive integer \(n\), the reciprocal of \(n\) raised to the power of 
\(s\) is summed. When the real part of \(s\) is greater than 1, the series converges to a finite value. However, for other values of 
\(s\), including complex numbers, the series may not converge, and the Riemann zeta function takes on more interesting properties.


In 1859, Riemann published a groundbreaking paper presenting a precise formula for calculating the number of prime numbers up to any 
given limit. This formula represented a significant improvement over the previous approximate estimation provided by the prime number 
theorem. However, Riemann's formula relied on determining the values at which a generalized version of the zeta function equals zero. 
(The Riemann zeta function is defined for all complex numbers of the form x + iy, where i represents the square root of -1, except 
for the line x = 1.) Riemann discovered that the function equals zero for all negative even integers such as -2, -4, -6, etc., which 
are referred to as "trivial zeros." Furthermore, he observed an infinite number of zeros within the critical strip of complex numbers 
lying between the lines x = 0 and x = 1. Additionally, Riemann noticed that all the nontrivial zeros exhibit symmetry with respect to 
the critical line x = 1/2. Based on these findings, Riemann proposed a conjecture, now famously known as the Riemann hypothesis, 
suggesting that all nontrivial zeros of the zeta function lie exactly on the critical line. 

The Riemann zeta function has deep connections to prime numbers and the distribution of prime numbers, as well as to the distribution 
of zeros of the function in the complex plane, which is related to the unsolved Riemann Hypothesis.

\subsection{Euler's Contribution}
\begin{equation}
    \zeta(s) =\prod_{p=prime} \frac{1}{1-p^-s}
\end{equation}

It was with this relation that the intimate connection of the Zeta function with the primes was discovered. Cracking the Riemann 
Hypothesis would provide a major breakthrough in identifying and proving a lot of properties related to prime numbers

\subsection{Reference}

The equation and its description were referenced from the following sources:

\footnote{\url{https://en.wikipedia.org/wiki/Riemann_zeta_function}}
\footnote{\url{https://www.britannica.com/science/Riemann-zeta-function}}

\end{document}

